 \documentclass[aps,prl,twocolumn,showpacs,showpacs,10pt,superscriptaddress]{revtex4-1}

%\linespread{1.5}
\usepackage[T1]{fontenc}
\usepackage{mathptmx}
\usepackage{graphicx}
\usepackage{subfigure}
\usepackage{amsmath}
\usepackage{amsfonts}
\usepackage{amssymb}
\usepackage{mathrsfs}
\usepackage{bbm}
\usepackage{subfigure}
\usepackage{xcolor}
\usepackage[colorlinks=true]{hyperref}
\usepackage{ulem}

\newcommand{\red}[1]{\textcolor{red}{#1}}
\newcommand{\blue}[1]{\textcolor{blue}{#1}}
\newcommand{\green}[1]{\textcolor{green}{#1}}
\newcommand{\gray}[1]{\textcolor{gray}{#1}}
\newcommand{\orange}[1]{\textcolor{orange}{#1}}


\newlength{\seplinewidth}
\newlength{\seplinesep}
\setlength{\seplinewidth}{1mm}
\setlength{\seplinesep}{2mm}
\colorlet{sepline}{orange}
\newcommand*{\sepline}{%
  \par
  \vspace{\dimexpr\seplinesep+.5\parskip}%
  \cleaders\vbox{%
    \begingroup % because of color
      \color{sepline}%
      \hrule width\linewidth height\seplinewidth
    \endgroup
  }\vskip\seplinewidth
  \vspace{\dimexpr\seplinesep-.5\parskip}%
}








\begin{document}

\title{Landau-Zener Effect Induced Hysteresis in Topological Josephson Junctions}


\author{Jia-Jin Feng}
\altaffiliation{These authors contributed equally to this work.}
\affiliation{School of Physics, Sun Yat-sen University, Guangzhou 510275, China}

\author{Zhao Huang}
\altaffiliation{These authors contributed equally to this work.}
\affiliation{Texas Center for Superconductivity, University of Houston, Houston, Texas 77204, USA}

\author{Zhi Wang}
\email{wangzh356@mail.sysu.edu.cn}
\affiliation{School of Physics, Sun Yat-sen University, Guangzhou 510275, China}
\affiliation{Department of Physics, The University of Texas at Austin, Austin, Texas 78712, USA}

\author{Qian Niu}
\affiliation{Department of Physics, The University of Texas at Austin, Austin, Texas 78712, USA}


\begin{abstract}
We reveal that topological Josephson junctions provide a natural platform for the interplay between the Josephson effect and LZ effect through a two-level system formed by coupled Majorana modes. We build a quantum resistively shunted junction model (QRSJ) by including the quantum two levels into the standard RSJ model. We obtain hysteresis in the I-V characteristics which exists in the overdamped regime while the hysteresis is not available when the LZ effect is absent. We also demonstrate a mapping from the quantum dynamics to a purely classical nonlinear dynamics in a restricted three-dimensional phase space, where the hysteresis can be understood as coming from the different static and kinetic dry friction. Further investigations in a corresponding topological SQUID reveal interference patterns with periods $h/e$ and $h/2e$ for the switching and retrapping current respectively.
\end{abstract}
\date{\today}
\pacs{74.50.+r, 03.65.Sq, 85.25.Dq, 74.78.Na}
\maketitle

%74.50.+r (Tunneling phenomena; Josephson effects),
%03.65.-w (Quantum mechainics),
%03.67.-a (Quantum information),
%74.90.+n (Other topics in superconductivity),
%74.81.Fa (Josephson junction arrays and wire networks),
%76.63.Nm (Quantum wires),
%74.20.Mn (Nonconventional mechanisms),
%85.25.Cp (Josephson devices), 85.25.Dq (SQUIDs)}

{\it Introduction.---}
The topologically protected degeneracy related to nonlocal nature of Majorana zero modes (MZMs) is among the core features of topological superconductors \cite{kitaev01,kitaevaip,zhangrmp}. This degeneracy is the foundation of fascinating topological qubits \cite{fu08,sato09,Tanaka09,sauprl10,alicea12,beenakker13,franzrmp,aliceaprx,aguadoreview} and also related to supersymmetry in the 0+1 dimension \cite{Qi09,Hsieh16,Huang17}. The situation is interesting as well when the degeneracy is split by couplings between MZMs \cite{beenakker08, Cheng09, Muzushima10,tewarijpcm,marcus16}. In particular for the one-dimensional case \cite{fuprb09,Lutchyn10,oregprl10,Mourik12,Deng12}, the split energy levels form a typical two-level system (TLS) since other excitation levels have much higher energy \cite{alicea12,Platero12,aguado11}.

\begin{figure}
\begin{center}
\includegraphics[clip = false, width = 1 \columnwidth]{setup.pdf}
\caption{(a) Schematic of a topological Josephson junction driven by an injected current $I$. The single-electron tunneling through the Majorana zero modes and the Cooper-pair tunneling induce Josephson couplings quantified by energy scales of $E_{\rm M}$ and $E_J$ respectively. The junction has a resistance $R$ and a small capacitance $C$. (b) Energies of the TLS defined by the two Majorana zero modes touching the junction, where the Landau-Zener transition happens at the avoided energy crossing with $P$ the possibility of the system to stay at the higher energy level after the LZ transition. (c) Schematic of equivalent electric circuit for topological Josephson junction with both conventional Cooper-pair tunneling and single-electron tunneling through MZMs, shunted by R and C.}
\label{fig:setup}
\end{center}
\end{figure}


The two-level systems (TLSs) with their energy difference in control have proved extraordinarily fertile for interesting quantum phenomena \cite{AllenBook,Chuang05,Morsch06,noripr}. By coupling two MZMs with a Josephson junction as in Fig. \ref{fig:setup}a, two levels with energies $E\propto\pm\cos\theta/2$ are obtained and the plus/minus signs correspond to states with opposite fermion number parity respectively. Either level can coherently transport one electron through the junction, leading to the fractional Josephson effect $I \propto \pm \sin \theta/2$ \cite{fuprb09,Lutchyn10,oregprl10}. In realistic systems where the two levels are inevitably coupled, the TLS has avoided level crossings at $\theta = (2n+1)\pi$ as in Fig. \ref{fig:setup}b. Energy spectra with such avoided crossings are well known for the existence of the Landau-Zener (LZ) transitions \cite{LZ}: the TLS enters a superposition state when the phase difference is driven by a finite voltage drop across the junction \cite{wangpra}. The topological Josephson junction thus hosts a natural platform for the interplay between the LZ effect and Josephson effect. Since LZ effect has proved its impact on qualitatively changing the dynamics in various systems \cite{Chen11,Liu13,ludwig,higuchi,Law16,wubiao}, novel phenomena stemming from this interplay are expected on the topological junctions.



In this letter, we study the I-V characteristics of a realistic topological Josephson junction as sketched in Fig. \ref{fig:setup}a, where the supercurrent is contributed by tunneling in the form of both the single electron and Cooper pair. To obtain the universal feature of transport properties, we build a quantum resistively and capacitively shunted junction (QRCSJ) model which includes the extra quantum degree of freedom of the two levels. By injecting a dc current, we study two different adiabatic processes: increasing the current from zero to passing a switching current for the voltage up to nonzero and decreasing a large current to passing a retrapping current for the voltage down to zero. Since the LZ transitions need a finite voltage to rotate the phase, the LZ effect is available in the current-decreasing process but absent in the current-increasing process. The consequence is the different values of the switching and retrapping current, which further leads to nontrivial hysteresis in I-V characteristics. In particular, this hysteresis exists even in the overdamped regime, in contrast to its absence in conventional Josephson junctions without MZMs. We further map this quantum mechanical model to a three-dimensional pure classical mechanical model, investigating the trajectory of the motion in the three dimensional phase space and showing that the hysteresis is equivalent to the mechanical hysteresis from dry friction.



{\it Model.---}
The topological Josephson junction sketched in Fig. \ref{fig:setup}a consists of a tunneling barrier between two topological superconductors, which could be one-dimensional nanowires with spin-orbit couplings \cite{marcus16}, superconducting quantum spin-hall edge states \cite{kouwenhoven15}, or ferromagnetic atomic chains \cite{yazdani14}. They share the same effective model proposed by Kitaev which, for the low-energy physics, involves two MZMs $\gamma_{\rm L}$ and $\gamma_{\rm R}$ \cite{kitaev01}. Their coupling has the form \cite{fuprb09}
\begin{eqnarray}\label{eq:MJE}
\mathcal{H}_{\rm M} = -i E_{\rm M} \gamma_{\rm L}\gamma_{\rm R}\cos({\theta}/{2})
\end{eqnarray}
with $\theta$ the superconducting phase difference across the tunneling barrier and $E_{\rm M}$ the maximum coupling energy between MBSs. By defining a Dirac fermion $f = \gamma_{\rm L} + i \gamma_{\rm R}$, the Hamiltonian describes a typical TLS where the empty state $|0\rangle$ and occupied state $|1\rangle$ are the two eigenstates. The corresponding energy spectra are $E_{\pm} = \pm E_{\rm M} \cos({\theta}/{2})$ which cross at $\theta = (2n+1)\pi$ due to $E_{+}(\theta) = E_{-}(\theta + 2\pi)$. In realistic materials with finite size, the inevitable overlapping between edge Majorana modes lead to hybridization of the two states, which breaks the local parity conservation and produces avoided energy crossings as shown in Fig. \ref{fig:setup}b. By writing the state as $|\psi \rangle = \psi_0  |0\rangle +  \psi_1  |1\rangle$, the dynamics of this TLS is determined by the Schr\"{o}dinger equation
\begin{eqnarray}\label{eq:SE}
i \frac{d}{dt} \left(\begin{array}{cc}
\psi_0  \\
\psi_1
\end{array}\right) = \left(\begin{array}{cc}
E_{\rm M}\cos{\frac{\theta}{2}} & \delta\\
\delta & -E_{\rm M}\cos{\frac{\theta}{2}}
\end{array}\right) \left(\begin{array}{cc}
\psi_0  \\
\psi_1
\end{array}\right)
\end{eqnarray}
with $\delta$ the effective coupling between two states with opposite parities, which could be derived from realistic models \cite{supplement}. When the motion of $\theta$ is driven by a finite voltage drop across the junction, the Landau-Zener transitions take place and the system has finite possibilities to stay on both the energy levels.

To solve Eq. (2), we need to learn the motion of $\theta$, which can be generally described by the RCSJ model \cite{tinkhambook}, where the total injected current $I$ is divided by the resistive, capacitive and Josephson channels as $I= C {dV(t)}/{dt} +{V(t)}/{ R } + I_{\rm J}(t)$ with $C$ the capacitance, $V(t)$ the instantaneous voltage bias and $R$ the resistance, as shown schematically in Fig. \ref{fig:setup}c. Here we consider the realistic cases in experiments where $I_{\rm J}$ has two parts: the conventional channel with a simple sine function $I_1(t) = I_{\rm c1}\sin{\theta(t)}$ and the parity dependent Majorana channel $I_2(t)  =  I_{\rm c2}\sin{\frac{\theta(t)}{2}}\langle \psi| i\gamma_2\gamma_3 |\psi\rangle$ with the maximum current $I_{\rm c2}={eE_{\rm M}}/{\hbar}$ and the quantum average $\langle \psi | i \gamma_2\gamma_3 |\psi\rangle = |\psi_1(t)|^2 - |\psi_0(t)|^2$. By invoking the ac Josephson relation $ d \theta/dt = 2eV/\hbar$, we obtain $I(t) = C_0{d^2\theta}/{dt^2} + R_0^{-1}{d\theta}/{dt} + I_1(t) + I_2 (t)$ with $C_0 =  \hbar^2 C  /  4e^2$ and $R_0 = 2e R / \hbar$. We consider the overdamped junction regime where the capacitance is negligibly small. This equation can be rewritten explicitly as
\begin{equation}\label{eq:RSJ}
\frac{d\theta}{dt} = R_0 \left[I - I_{\rm c1}\sin{\theta} - I_{\rm c2}\left(|\psi_1|^2 - |\psi_0|^2\right)  \sin{\frac{\theta}{2}}\right].
\end{equation}


The coupling between Eq.~(\ref{eq:SE}) and (\ref{eq:RSJ}) constitutes the QRSJ model which provides the opportunities for the interplay between Landau-Zener effect and Josephson effect.
The Schr\"{o}dinger equation (\ref{eq:SE}) is a standard equation for Landau-Zener transitions where the dynamics of wave functions depends on the Josephson phase $\theta$. The motion of $\theta$, in turn, is determined by the wave functions as seen at the right hand side of Eq.~(\ref{eq:RSJ}). In this way, the Josephson phase correlates with the wave function of the TLS, and they together determine the quantum dynamics and transport properties of the junction. Considering the absence of such interplay in conventional Josephson junctions, novel transportation phenomena are expected in this system.




{\it Landau-Zener Effect Induced Hysteresis.---}
Now we study the I-V characteristics of the topological Josephson junction based on the QRSJ model. We numerically simulate the average voltage upon adiabatic current injection, which gradually increases to a large value and then decreases back to zero.
As a benchmark, we first show the I-V curve for a trivial junction with $I_{\rm c2} = 0$ in Fig. \ref{fig:qrcsj}a, which is the well known result of $V= R \sqrt{ I^2 - I^2_{\rm c1}}$ around the critical current \cite{tinkhambook}. We then consider an additional $4\pi$-period Josephson current $I_2=I_{\rm c2}\sin \frac{\theta}{2}$ which corresponds to the case of local parity conservation with $\delta = 0$, where LZ effect cannot take place. We solve the Eq.~(\ref{eq:RSJ}) with $|\psi_0|^2=1$ or $|\psi_1|^2=1$, and obtain the I-V curve as shown in Fig.~\ref{fig:qrcsj}b. Clearly the simple addition of a $4\pi$-period Josephson current modifies the shape of the I-V curve but demonstrates no novel phenomenon. For both cases, the voltage which is the velocity of the phase difference is fully determined by the applied current, so the quantum dynamics is history independent. 

However, when $\delta$ becomes finite and the Landau-Zener transitions begin to affect the tunneling current, we find an unambiguous hysteretic I-V curve with two critical currents as shown in Fig. \ref{fig:qrcsj}c: a switching current $I_{\rm sw}$ where the voltage jumps from zero to a finite value, and a smaller retrapping current $I_{\rm re}$ for the finite voltage jumping back zero.


\begin{figure}[t]
\begin{center}
\includegraphics[clip = true, width = 1 \columnwidth]{qrcsj.pdf}
\caption{I-V curves in absence of LZ effect for (a) $I_{\rm c2} = 0$, (b) $I_{\rm c2}/I_{\rm c1} = 2$ and $\delta=0$. (c) I-V curves in presence of LZ effect with $I_{\rm c2}/I_{\rm c1} = 2$ and $\delta/E_{\rm M} = 0.02$. (d) The time evolution of the quantum state of the two-level system when changing $I$ from $I<I_{\rm sw}$ to $I>I_{\rm sw}$ with $|\psi_0|^2$ in red and $|\psi_1|^2$ in blue. The resistance is taken as $R_0 = 10$.}
\label{fig:qrcsj}
\end{center}
\end{figure}

To reveal the origin of this hysteresis, in Fig. \ref{fig:qrcsj}d we show the time evolution of $|\psi_0|^2$ and $|\psi_1|^2$ when the injected current increases across the switching point. Initially for a small injected current below the switch value, the voltage is zero and the TLS stays at $|1\rangle$ with certainty ($|\psi_1|^2=1$). When the injected current is increased above the switching current, the probabilities begin to oscillate. This is the St\"{u}ckelburg interference between multiple LZ transitions which is driven by the voltage drop across the junction \cite{LZ,noripr}. The oscillation is strongly damped because the classical resistance couples to this quantum mechanical system as a damping factor \cite{tinkhambook}. After a while, the TLS enters a state with nearly equal probability of the two levels since they are symmetric with the phase translation.

From above, we can see that the Josephson current is contributed by only one level for the zero-voltage stage but both levels for the finite-voltage stage. Therefore, it is reasonable that the critical currents are different when the injected current is increasing or decreasing. Because only one level contributes to the Josephson current as in the current increasing stage, we have $I_{\rm J}=I_{\rm c1}\sin{\theta}+I_{\rm c2}\sin{\frac{\theta}{2}}$ and the critical current is
\begin{eqnarray}\label{eq:Isw}
I_{\rm sw} = (2I_{\rm c1}  \zeta  + I_{\rm c2}   ) \sqrt{1- \zeta^2},
\end{eqnarray}
with $\zeta =\sqrt{I_{\rm c2}^2/I_{\rm c1}^2 + {1}/{2}} - I_{\rm c2}/I_{\rm c1}$.
On the other hand in the current decreasing stage, because the TLS has finite probabilities on both levels due to the Landau-Zener transitions, the Josephson current changes to $I_{\rm J}=I_{\rm c1}\sin{\theta}+I_{\rm c2}(|\psi_1|^2 - |\psi_0|^2)\sin{\frac{\theta}{2}}$. For this case, the critical current would be smaller, since the two levels with opposite parities carry opposite currents and cancel each other. If the cancellation is perfect with $|\psi_1| = |\psi_0|$, the corresponding critical current is
\begin{equation}\label{eq:Ire}
I_{\rm re} = I_{\rm c1}.
\end{equation}
We notice that the numerical results for $I_{\rm re}$ indeed agree with this critical value quite well by comparing the value of critical currents in Fig. \ref{fig:qrcsj}a and Fig. \ref{fig:qrcsj}c.


The above study provides a new hysteresis phenomenon in condensed matter physics. We note that this phenomenon involves neither local nor global parity conservation and is immune to various quasiparticle poisoning effects in realistic setups \cite{supplement,yangpeng}. Therefore, its measurement is feasible for a variety of candidate topological Josephson systems.



{\it Mapping to classical nonlinear dynamics.---}
The QRSJ model which is described by the coupled quantum equation (\ref{eq:SE}) and classical equation (\ref{eq:RSJ}) can be mapped to a purely classical model of nonlinear dynamics. The trick is to cast the TLS into a classical Hamiltonian, which has been shown to be useful for understanding the nonlinear Landau-Zener problem \cite{Liu02,Liu03}. To do so, we note that despite the wave function of the TLS $(\psi_0,\psi_1)$ seemingly has four variables, they are subjected to two restrictions from the quantum mechanics: the wave function must be normalized and the total phase of the wave function is irrelevant. Then we can define two canonical variables $s = |\psi_1|^2 - |\psi_0|^2$ and $\phi = {\rm arg}\psi_1 - {\rm arg}\psi_0$, which are able to completely capture the physics of the TLS. With this technique, we can cast the QRSJ model to a purely classical model by defining a generalized potential function,
\begin{equation}
U (s,\phi,\theta)= -s E_{\rm M}  \cos \frac{\theta}{2} + \delta \sqrt{1-s^2} \cos \phi  -E_{\rm J} \cos \theta - \frac{\hbar}{2e} I \theta.
\end{equation}
Three equations can be derived from this generalized potential,
\begin{subequations}\label{eq:classical}
\begin{align}
&\frac{d\theta}{dt} = - R \frac {\partial U}{\partial \theta}  =  R_0 \left[I - I_{\rm c1}\sin{\theta} - s I_{\rm c2}  \sin{\frac{\theta}{2}}\right] ,
\\
&\frac{ds}{dt} = \frac{1}{\hbar} \frac {\partial U}{\partial \phi} = -\frac{\delta}{\hbar} \sqrt{1-s^2} \sin \phi ,
\\
& \frac{d\phi}{dt} = - \frac{1}{\hbar} \frac{\partial U}{ \partial s} =  \frac{E_{\rm M}}{\hbar}\cos \frac{\theta}{2} + \frac{s \delta }{\hbar \sqrt{1-s^2}}  \cos \phi.
\end{align}
\end{subequations}
Obviously the Eq. (\ref{eq:classical})a is identical to Eq. (\ref{eq:RSJ}), while Eqs. (\ref{eq:classical})b and (\ref{eq:classical})c are equivalent to Eq. (\ref{eq:SE}) which could be verified through some simple algebra \cite{supplement}. Thus we successfully transform the problem for a topological junction to the motion of a classical particle in a three dimensional space.



In this pure classical model, we analyze the dynamical stability of the junction with the injected current $I$ as the control parameter. We numerically study the trajectory of the particle with different control parameters and initial conditions. As shown in Fig. \ref{fig:classical}, we explore three different parameter regimes. For the small control parameters of $I< I_{\rm re}$, the trajectories for all initial conditions are closed, demonstrating circles in the $s-\phi$ plane as seen in Fig. \ref{fig:classical}a. For intermediate parameters of $I_{\rm re}<I<I_{\rm sw}$, there are two different types of trajectories, depending on the initial conditions. The trajectory for a large initial $s$ is closed while the trajectory for a small initial $s$ is not closed, falling to $s \approx 0$ instead. For large parameters of $I> I_{\rm sw}$, all trajectories are falling to $s \approx 0$. These results demonstrate that the dynamics in the intermediate regime of $I_{\rm re} <I<I_{\rm sw}$ depends on the initial value of $s$. This history dependence is the key to the hysteresis.



\begin{figure}[t]
\begin{center}
\includegraphics[clip = true, width = 1 \columnwidth]{classical.pdf}
\caption{Typical trajectories of the particle in the phase space for the injected current of (a) $I/I_{c1} = 0.5$ below the retrapping current, (b) $I/I_{c1} = 2.2 $ between the retrapping current and the switching current, and (c) $I/I_{c1} = 4$ above the switching current. Other parameters are taken the same as in Fig.~\ref{fig:qrcsj}. In panel (b), the shape depends on the initial condition: large $s$ or small $s$.}
\label{fig:classical}
\end{center}
\end{figure}

In this classical model, the hysteresis is similar to the mechanical hysteresis from the dry friction \cite{wojewoda} since Eq. (\ref{eq:classical}a) is actually a friction equation. That is, the particle has different friction forces when it is static and moving in the direction of $\theta$. This difference comes from the history dependent trajectories in the $s-\phi$ plane, as shown in Fig. \ref{fig:classical}, and then feedback to the motion in the $\theta$ direction through the last term in Eq. (\ref{eq:classical})a. This feedback effectively induces a difference in the static friction and dynamic friction for the particle; therefore the particle would begin and stop moving at different dragging forces.


{\it Interference Pattern of Topological SQUID.---}
Let us now consider a topological SQUID which contains four MZMs as shown in Fig. \ref{fig:squid}a. The same as for the single topological junction, the I-V curve of this SQUID should also be hysteretic. Then we expect two interference patterns of maximum supercurrent, one for the switching current and the other for the retrapping current. The switching current should contain contributions from both the conventional and MZM channel and is thus given by
\begin{eqnarray}
I_{\rm sw} (\Phi) = \max_\theta  \big[I_{\rm J1} \sin \theta &+& I_{\rm J2} \sin (\theta + \frac{2\pi \Phi}  {\Phi_0} ) \\\nonumber
                      + I_{\rm M1}  \sin \frac{\theta}{2} &+& I_{\rm M2} \sin (\frac{\theta}{2}+\frac{\pi \Phi}{\Phi_0})\big],
\end{eqnarray}
where $I_{{\rm J}1,{\rm J}2,{\rm M}1,{\rm M}2}$ represent the critical currents of different channels, $\Phi$ is the magnetic flux through the SQUID, and $\Phi_0 = h/2e$ is the superconducting flux quantum. Here we require the parity conservation of the coupled MZMs.
This interference pattern, as shown explicitly by the yellow solid line in Fig.~\ref{fig:squid}b, is obviously $2\Phi_0$-periodic, which agrees with previous studies \cite{beenakker11,veldhorst12}.
On the other hand, the currents from Majorana channels are almost canceled when considering the retrapping current, which leads to
\begin{eqnarray}
I_{\rm re} (\Phi) \approx \max_\theta\big[I_{\rm J1} \sin \theta + I_{\rm J2} \sin (\theta + 2\pi \Phi / \Phi_0) ],
\end{eqnarray}
which is $\Phi_0$-periodic as shown by the blue solid line in Fig. \ref{fig:squid}b.
$I_{\rm sw}$ and $I_{\rm re}$ can be directly obtained by numerically studying the dynamics with the RSJ model, where the Hamiltonian for the coupled Majorana modes in the SQUID is $H= - i\gamma_1 \gamma_4 E_{\rm u} \cos {\theta}/{2}
-  i\gamma_2 \gamma_3 E_{\rm d} \cos {(\theta+ 2\pi\Phi/\Phi_0)}/{2}
+ i\delta_{\rm l} \gamma_1 \gamma_2 + i\delta_{\rm r} \gamma_3 \gamma_4$,
with $E_{\rm u,d}$ and $\delta_{\rm l,r}$ the corresponding coupling coefficients. The numerical results are shown in Fig. \ref{fig:squid}b, which agree well with our analytical results.

\begin{figure}[t]
\begin{center}
\includegraphics[clip = true, width = 0.95 \columnwidth]{squid.pdf}
\caption{(a) Schematic setup of a topological SQUID structure with four Majorana zero modes. (b) The analytical interference pattern for the switching current (blue solid line) and the retrapping current (orange solid line), and the numerically results for the interference patterns of switching current (blue circle) and retrapping current (orange diamond). Parameters are taken the same as in Fig. \ref{fig:qrcsj}.}
\label{fig:squid}
\end{center}
\end{figure}

From both the analytical and numerical results, in a topological SQUID we can obtain coexistence of $h/e$ and $h/2e$-periodic interference patterns, which as far as we know is never seen in any SQUID before.

{\it Discussions and Conclusion.---}
 Experimental verification of our theoretical proposal is optimistic since
measuring the I-V curve is a routine experiment in Josephson junctions. Recently, unexpected hysteresis behavior has already been reported in a number of overdamped topological insulator junctions \cite{molenkamp13,molenkamp16}. We also notice that a recent experiment observed $h/e$-periodicity in a quantum spin Hall insulator SQUID, where hysteresis is also observed in the same experiment \cite{kouwenhoven15}. Our theory shows a path to examine the relevance of these experimental results to the existence of MZMs.

The RSJ model is often understood with a tilted washboard picture \cite{tinkhambook}. The QRSJ model can also be included in this picture, but with an extra "spin" degree of freedom. In this picture, $\theta$ can be viewed as the coordinate of a classical particle and the RSJ equation is analogous to the Newtonian equation of motion for the particle in an effective potential with a Zeeman term, and under a viscous force. Therefore, the QRSJ model in this system corresponds to the dynamics of a spinful semiclassical particle in the tilted-washboard potential, where the parity serves as the pseudo-spin.

In above, we have mainly studied overdamped junctions which have negligible capacitance. For underdamped junctions which have non-negligible capacitance, the hysteresis is also available in the trivial topological regime \cite{tinkhambook}. However, the difference between the switching and retrapping current should be less than the difference in topological nontrivial regime by considering the contribution of currents from the Majorana channels \cite{supplement}.


In summary, we propose that the Landau-Zener effect of the two-level system in an topological Josephson junction can lead to hysteresis in the I-V characteristics, which exists even for overdamped junctions. We also demonstrate coexistence of a $h/e$-periodic interference pattern for the switching current and a $h/2e$-periodic retrapping current in a topological SQUID. We map the quantum mechanical dynamics in topological Josephson junctions to a classical nonlinear dynamics, which introduces a new way to understand and analyze the dynamics. This mapping enables the study of topological Josephson effect with the tools in classical mechanics, and also suggests possibilities to dig out more nontrivial phenomena which have counterparts in nonlinear physics as future works.

\acknowledgments
{\it Acknowledgments.---}
The authors are grateful for Pavan Hosur, Stefan Ludwig and Hongqi Xu for helpful discussions. This work was supported by the National Natural Science Foundation of China under Grants No. 11774435 and No. 61471401, and China Scholarship Council under Grants No. 201706385057. Zhao Huang is supported by Robert A. Welch Foundation under Grant No. E-1146.

\begin{thebibliography}{00}

\bibitem{kitaev01} A. Y. Kitaev, Phys. Usp. \textbf{44}, 131 (2001).

\bibitem{kitaevaip}	A. Kitaev, AIP Conference Proceedings \textbf{1134}, 22 (2009).

\bibitem{zhangrmp} X. L. Qi and S. C. Zhang, Rev. Mod. Phys. \textbf{83}, 1057 (2011).

%%%%%%%%%%%%%%

\bibitem{fu08}	L. Fu and C. L. Kane, Phys. Rev. Lett. \textbf{100}, 096407 (2008).

\bibitem{sato09} M. Sato, Y. Takahashi, and S. Fujimoto Phys. Rev. Lett. \textbf{103}, 020401 (2009).

\bibitem{Tanaka09} Y. Tanaka, T. Yokoyama, and N. Nagaosa, Phys. Rev. Lett. \textbf{103}, 107002 (2009).

\bibitem{sauprl10} J. D. Sau, R. M. Lutchyn, S. Tewari, and S. Das Sarma, Phys. Rev. Lett. \textbf{104}, 040502 (2010).

\bibitem{alicea12} J. Alicea, Rep. Prog. Phys \textbf{75}, 076501 (2012).

\bibitem{beenakker13} C.W.J. Beenakker, Annu. Rev. Con. Mat. Phys. \textbf{4}, 113 (2013)

\bibitem{franzrmp} S. R. Elliott and M. Franz, Rev. Mod. Phys. \textbf{87}, 137 (2015).

\bibitem{aliceaprx}	D. Aasen, M. Hell, R. V. Mishmash, A. Higginbotham, J. Danon, M. Leijnse, T. S. Jespersen, J. A. Folk, C. M. Marcus, K. Flensberg, and J. Alicea, Phys. Rev. X \textbf{6}, 031016 (2016).


\bibitem{aguadoreview} R. Aguado, Riv. Nuovo Cimento \textbf{11}, 523 (2017).


\bibitem{Qi09} Xiao-Liang Qi, Taylor L. Hughes, S. Raghu, and Shou-Cheng Zhang, Phys. Rev. Lett. \textbf{102}, 187001 (2009).

\bibitem{Hsieh16} T. H. Hsieh, G. B. Hal\'asz, and T. Grover, Phys. Rev. Lett. \textbf{117}, 166802 (2016).

\bibitem{Huang17} Z. Huang, S. Shimasaki, and M. Nitta Phys. Rev. B \textbf{96}, 220504(R) (2017).








%%%%%%%%%%

\bibitem{beenakker08}	J. Nilsson, A. R. Akhmerov, and C. W. J. Beenakker, Phys. Rev. Lett. \textbf{101}, 120403 (2008).

\bibitem{Cheng09} Meng Cheng, Roman M. Lutchyn, Victor Galitski, and S. Das Sarma Phys. Rev. Lett. \textbf{103}, 107001 (2009).

\bibitem{Muzushima10} T. Mizushima and K. Machida, Phys. Rev. A \textbf{82}, 023624 (2010).

\bibitem{tewarijpcm} T. D. Stanescu and S. Tewari, J. Phys.: Condens. Matter \textbf{25}, 233201 (2013).

\bibitem{marcus16} S. M. Albrecht, A. P. Higginbotham, M. Madsen, F. Kuemmeth, T. S. Jespersen, J. Nyg{\aa}rd, P. Krogstrup, and C. M. Marcus, Nature \textbf{531}, 206 (2016).

\bibitem{prada17} J. Cayao, P. San-Jose, A. M. Black-Schaffer, R. Aguado, and E. Prada, Phys. Rev. B \textbf{96}, 205425 (2017).

\bibitem{fuprb09} L. Fu and C. L. Kane, Phys. Rev. B \textbf{79}, 161408 (2009).

\bibitem{Lutchyn10} R. M. Lutchyn, J. D. Sau, and S. Das Sarma, Phys. Rev. Lett. \textbf{105}, 077001 (2010).

\bibitem{oregprl10} Y. Oreg, G. Refael, and F. von Oppen, Phys. Rev. Lett. \textbf{105},
177002 (2010).

\bibitem{Mourik12} V. Mourik, K. Zuo, S. M. Frolov, S. R. Plissard, E. P. A. M. Bakkers, and L. P. Kouwenhoven, Science \textbf{336}, 1003 (2012).

\bibitem{Deng12} M. T. Deng, C. L. Yu, G. Y. Huang, M. Larsson, P. Caro, and H. Q. Xu, Nano Lett. \textbf{12}, 6414 (2012).

\bibitem{Platero12} F. Dominguez, F. Hassler, G. Platero Phys. Rev. B \textbf{86}, 140503(R) (2012)

\bibitem{aguado11} P. San-Jose, E. Prada, and R. Aguado, Phys. Rev. Lett. \textbf{108}, 257001 (2012).

%%%%%%%%%%%%%%%%%
\bibitem{AllenBook} L. Allen, and J. H. Eberly (1974), {\it Optical Resonance and Two-level Atoms} (Dover, 1975).

\bibitem{Chuang05} L. M. K. Vandersypen and I. L. Chuang, Rev. Mod. Phys. \textbf{76}, 1037 (2005).

\bibitem{Morsch06} O. Morsch and M. Oberthaler, Rev. Mod. Phys. \textbf{78}, 179 (2006).

\bibitem{noripr} S. N. Shevchenko, S. Ashhab and F. Nori, Phys. Rep. \textbf{492}, 1 (2010).






%%%%%%%%%%%%%%%%%%%%




%%%%%%%%%%%%%%


\bibitem{LZ} L. Landau, Phys. Z. Sowjetunion, \textbf{2}, 46 (1932); C. Zener, Proc. R. Soc. Landon Ser. A \textbf{137} 696(1932); E. C. G. St\"ueckelberg, Helv. Phys. Acta \textbf{5}, 369 (1932).


\bibitem{wangpra} W. C. Huang, Q. F. Liang, D. X. Yao and Z. Wang, Phys. Rev A \textbf{92}, 012308 (2015).

\bibitem{wubiao} B. Wu and Q. Niu, Phys. Rev. A \textbf{61}, 023402 (2000).

\bibitem{Chen11} Y. A. Chen, S. D. Huber, S. Trotzky, I. Bloch, and E. Altman, Nature Physics \textbf{7}, 61 (2011).

\bibitem{Liu13} X. J. Liu, K. T. Law, T. K. Ng, and P. A. Lee, Phys. Rev. Lett. \textbf{111}, 120402 (2013).

\bibitem{ludwig} F. Forster, G. Petersen, S. Manus, P. HANGGI, D. Schuh, W. Wegscheider, S. Kohler, and S. Ludwig, Phys. Rev. Lett. \textbf{112}, 116803 (2014).

\bibitem{Law16} W. Y. He, S. Z. Zhang, and K. T. Law, Phys. Rev. A \textbf{94}, 013606 (2016).

\bibitem{higuchi} T. Higuchi, C. Heide, K. Ullmann, H.B. Weber, and P. Hommelhoff, Nature \textbf{550}, 224 (2017).

\bibitem{kouwenhoven15} V. S. Pribiag, A. J. A. Beukman, F. Qu, M. C. Cassidy, C. Charpentier, W. Wegscheider, and L. P. Kouwenhoven, Nature Nanotechnology \textbf{10}, 593 (2015).

\bibitem{yazdani14} S. Nadj-Perge, I.K. Drozdov, J. Li, H. Chen, S. Jeon, J. Seo, A.H. MacDonald, B.A. Bernevig, and A. Yazdani, Science \textbf{346}, 602 (2014).

%%%%%%%%%%%%%%%%%%%
\bibitem{supplement} See Supplemental Material for derivation of the 2x2 Josephson Hamiltonian and the Josephson current, and the methods to cast a quantum two-level system into a classical system, and discussions of the external parity flipping, quasiparticle poisoning, and underdamped junction.
%%%%%%%%%%%%%%%%%%

\bibitem{tinkhambook}  M. Tinkham, {\it Introduction to Superconductivity}, (Second Edition, McGraw-Hill Book Co. 1996).

\bibitem{yangpeng} Y. Peng, F. Pientka, E. Berg, Y. Oreg, and F. von Oppen, Phys. Rev. B \textbf{94}, 085409 (2016).

%%%%%%%%%%%%%%%%%%%%%%%%%%%%%


\bibitem{Liu02} J. Liu, L. Fu, B. Y. Ou, S. G. Chen, D. I. Choi, B. Wu, and Q. Niu, Phys. Rev. A \textbf{66}, 023404 (2002).

\bibitem{Liu03} J. Liu, B. Wu, and Q. Niu, Phys. Rev. Lett. \textbf{90}, 170404 (2003).

\bibitem{wojewoda} J. Wojewoda, A. Stefanski, M. Wiercigroch, and T. Kapitaniak, Phil. Trans. R. Soc. A \textbf{366}, 747 (2008).

\bibitem{beenakker11} B. van Heck, F. Hassler, A. R. Akhmerov, and C. W. J. Beenakker, Phys. Rev. B \textbf{84}, 180502 (2011).

\bibitem{veldhorst12} M. Veldhorst, C. G. Molenaar, C. J. M. Verwijs, H. Hilgenkamp, and A. Brinkman, Phys. Rev. B \textbf{86}, 024509 (2012).


\bibitem{molenkamp13} J. B. Oostinga, L. Maier, P. Sch{\"{u}}ffelgen, D. Knott, C. Ames, C. Br{\"{u}}ne, G. Tkachov, H. Buhmann, and L. W. Molenkamp, Phys. Rev. X \textbf{3}, 021007 (2013).

\bibitem{molenkamp16} J. Wiedenmann, R. S. Deacon, S. Hartinger, O. Herrmann, T. M. Klapwijk, L. Maier, C. Ames, C. Br\"{u}ne, C. Gould, A. Oiwa, K. Ishibashi, S. Tarucha, H. Buhmann, L. W. Molenkamp, and E. Bocquillon, Nature Communications \textbf{7}, 10303 (2016). 




\end{thebibliography}





\end{document}
