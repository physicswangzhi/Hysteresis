\documentclass[onecolumn,preprintnumbers,amsmath,amssymb,prb]{revtex4}

\usepackage{graphicx}
\usepackage{bm}
\usepackage{subfigure}
\usepackage{color}

\newcommand{\red}[1]{\textcolor{red}{#1}}
\newcommand{\blue}[1]{\textcolor{blue}{#1}}

\begin{document}
%\tightenlines
%\nonum



\title{Reply to the comments}


\maketitle




\noindent Comment1: 
\blue{In the paper “Landau-Zener effect induced hysteresis in topological
Josephson junctions”, the authors study theoretically in detail the
hysteretic I-V curves for a topological Josephson junction. The
subject is obviously topical. But it is not convincing that the
results are interesting enough for the broad audience; they rather
look like quite technical ones, which may be interesting for
specialists. Also the presentation should be improved. To further
attract the attention of the authors to improving the paper, I
formulate below my particular comments, questions, and hesitations.}

\vspace{5mm}

\noindent Reply1: 
First, we thank the Referee for the positive evaluation on the subject of our work, and appreciate his valuable comments on the presentation of our results. We agree with the referee that our work is presented with too much details which may hurdle illustration the main results interesting to broad audience. Let us summarize the main results of our work in the following:

(a) We reveal that the quantum Landau-Zener transition plays an essential role in the topological Josephson junction. The interplay between the Landau-Zener effect and the topological Josephson effect qualitatively changes the dynamical and transport properties of the junction. 

(b) We establish a general quantum-classical coupled model (quantum RCSJ model) for study this interplay  of the Landau-Zener effect and the topological Josephson effect..

(c) As one example, we demonstrate a novel hysteresis which originate from the classical-quantum interplay.

(d) We show an interesting coexistence of $2\pi$ and $4\pi$ periodic 
interference patterns in the topological SQUID.


The results (a) and (b) are easily understandable for broad audience while the result (c) and (d) are a little bit technical, needing a detailed knowledge of the transport of Josephson junction. To accommodate with the comments of the referee, we completely rewrite the paper to sharpen our general results (a) and (b), while put the more technical results (c) and (d) in the secondary role.

The thank the Referee for the helpful point-to-point suggestions to improving the paper. In the following we answer the questions raised and elaborate the revisions made according to the suggestion of the Referee.




\red{TBD}

\vspace{5mm}

\noindent Comment2: 
\blue{The title tells that the hysteresis is induced by the LZ effect. This
is the nonadiabatic one-passage transition between the energy levels.
However, in the main text, in the paragraphs starting from the wording
“To reveal the origin of this hysteresis…”, it seems that it is
assumed there that this comes from the interference phenomena, “the
Stuckelberg interference". If this latter is correct, then many places
in the paper, the title including, should be changed.}

\vspace{5mm}

\noindent Reply2: We thank the Referee for this suggestion. We agree with this observation of the Referee. In this system, there are multiple passage of the energy levels and the Stuckelberg interference between the phase accumulated during the multiple passage is indeed crucial. In our understanding, the Stuckelberg interference is subject in the general study the Landau-Zener effect. We revise the title and the main body of the manuscript  according to this suggestion.  \red{TBD}

\vspace{5mm}

\noindent Comment3: 
\blue{In the abstract there is quite a strong statement: “Our results can
explain the unexpected hysteresis found in recent experiments.” But
this statement becomes unclear after reading the paper. If the
authors’ theory explains what was not explained in recent experiments,
this is important and should appear explicitly in both main text and
in detail in the supplementary. If it is only a speculation, and the
theory does not describe any specific experiment, then there is no
place for such a statement in the abstract. It is only in the
Discussions and Conclusions I found referring to experimental works
51, 52, saying that “Our theory shows a path to examine the relevance
of these experimental results to the existence of MZMs.” But this
seems to be a key issue of the paper, then why it is only mentioned
marginally, why there is no quantitative comparison with the
experiment, no detailed discussion?}


\vspace{5mm}

\noindent Reply3: We thank the Referee for this comment. Our theory is a general theoretical framework which does not describe any specific experiment. We only hope to point out that our theory could be used to understand the experimentally found hysteresis. We delete this statement in the abstract \red{TBD}.

\vspace{5mm}

\noindent Comment4: 
\blue{It seems to be incorrect to refer to the hysteresis as an “unexpected”
one, given that it was discussed and observed in several papers.}


\vspace{5mm}

\noindent Reply4:{We thank the Referee for this helpful comment. The meant the hysteresis is unexpected from the traditional theoretical point of view. We delete this word in the revised manuscript to avoid misunderstandings.}

\vspace{5mm}

\noindent Comment5: 
\blue{When discussing Fig. 4, the authors say about two periodicities. But
observed is only the total current not its two components. If so, then
only one periodicity will appear. How to extract from one single
observation the two periodicities? And again, since there are
experimental observations, why not to refer to them, to explain how to
visualize the two periodicities from particular experimental results?}

\vspace{5mm}

\noindent Reply5: We thank the Referee for the comment. \red{TBD [Do we need to tell editor that the referee did not understand.][Should we contact Molenkamp's group and refer our results to them?]}

\vspace{5mm}

\noindent Comment6: 
It seems that several important issues can be better related to
relevant literature:

* The relevance of the LZ transitions for Josephson junctions was
studied in many other works, including such early articles as PRL 75,
1831 (1995), PRL 81, 2538 (1998).

* Mapping to classical RSJ model. The analogy between quantum dynamics
of a JJ and its classical one was studied also before, e.g. Phys. Rep.
611, 1 (2016).

* Mapping of a quantum system to a classical two-level one was done
for diverse systems. For example, recently, on the example of two
mechanical resonators, e.g. D. Dragoman and M. Dragoman,
Quantum-Classical Analogies (Springer, 2004), Phys. Rev. A 94, 043855
(2016).

* The hysteresis and the QRSJ model, as in Fig. 1(c), seems to be
relevant for recent ideas about memory devices, e.g. Phys. Rev.
Applied 2, 034011 (2014), Phys. Rev. Applied 6, 014006 (2016), Sci.
Rep. 7, 46736 (2017). In all these works relevant was what the present
authors emphasize that “history dependence is the key to the
hysteresis”.

\vspace{5mm}


\noindent Reply6: We thank the Referee for this helpful suggestions. We added these papers in the revised reference. 

\vspace{5mm}

\noindent Comment7:
\blue{The value theta appears in the introduction and in Fig. 1 without
definition; it is rather defined only after Eq. (1).}

\vspace{5mm}


\noindent Reply7: We thank the Referee for this suggestion. We added the definition of $\theta$ in the place it is first introduced.


\vspace{5mm}

\noindent Comment8:
\blue{Does the Hamiltonian (1) describe both $I_{1}$ and $I_{2}$, which appear
below?}

\vspace{5mm}


\noindent Reply8: No, the Hamiltonian (1) only includes the contribution from the Majorana channel, which leads to $I_2$. The transitional Josephson current $I_1$ comes from the quasiparticle channels which can be describe by a Josephson energy $E = -E_J \cos \theta$. We added one sentence below Eq. (1) to elaborate this point.

\vspace{5mm}

\noindent Comment9:
\blue{The operators gammas, which appear in Eq. 1, are introduced in the
supplementary; what are they is not clear from reading the main text.}

\vspace{5mm}


\noindent Reply9: We thank the Referee for this suggestion. We added the introduction of the Majorana operator $\gamma$ below Eq. (1).




\vspace{5mm}

\noindent Comment10:
\blue{What is MBS (after Eq. 1)?}


\vspace{5mm}


\noindent Reply10: We apologize for the mistake of using abbreviations without definition. It should be "Majorana zero modes". We have corrected it in the revised manuscript.


\vspace{5mm}

\noindent Comment11:
\blue{Caption of Fig. 2: what are the units for $R_{0}$?}


\vspace{5mm}


\noindent Reply11: $R_0$ is a dimensionless number which is defined before Eq. (3). It represents the resistance of the junction $R$ in the units of $\hbar/2e$.


\vspace{5mm}

\noindent Comment12:
\blue{Ref. 33 includes only Landau, Zener and Stuckelberg, but the
Majorana’s paper is missing. Importantly (especially given the present
subject) that in his paper in 1932, Majorana also considered the
related phenomena, and obtained the very same LZ formula.}


\vspace{5mm}


\noindent Reply12: We thank the Referee for this helpful comment. We add the paper of Majorana in the reference.

\vspace{5mm}

\noindent Comment13:
\blue{In the paragraph before Eq. 1 capitalize “h” in “hall”.}


\vspace{5mm}


\noindent Reply13: We thank the Referee for his helpful comment. We make the revision accordingly.

\vspace{5mm}

\noindent Comment14:
\blue{After Eq. (S7) the authors say that they take the even total parity.
Why do they limit to this particular parity? As the next section says
the two parities and the flipping between them are relevant.}

\vspace{5mm}


\noindent Reply14: We thank the Referee for this comment. First, for the Majorana junction which is cleanly manufactured experimentally, the total parity is conserved. Therefore, even or odd total parity can be taken to simplify the formula. The flipping between even and odd parity, as we have shown in Sec. S3, is usually induced by external quantum states such as impurity state or accidental bound state. These external states may appear in some experimental samples, therefore we added a section in the supplementary section to discuss it. 


\vspace{5mm}

\noindent Comment15:
\blue{As far as I understand, $\phi_{0,1}$ are different from $\phi_{\alpha}$
above. This is somewhat misleading.}
\vspace{5mm}


\noindent Reply15: We thank the Referee for the suggestion. The symbols $\phi_{0,1}$ are indeed different from $\phi_{\alpha}$ with $\alpha = L,R$. For former represents the wave function of the two-level system, while the latter represents the superconducting phase difference in the left and the right superconductor. To avoid confusion, we change the $\phi_{\alpha}$ to $\theta_\alpha$. 


\vspace{5mm}

\noindent Comment16:
\blue{It is said that the total phase for the wave function has no physical
relevance. It is not exactly clear. In many cases, dynamical or
geometrical changes of wave functions result in the interference.}
\vspace{5mm}


\noindent Reply16: \red{TBD}



\vspace{5mm}

\noindent Comment17:
\blue{Please check the English of the paper. Some examples: “massage the
Schrodinger equation”, “the its own”, Hamilton instead of Hamiltonian
before and after Eq. S17, “We … obtains”}
\vspace{5mm}


\noindent Reply17: We thank the Referee for this comment. We thoroughly examined the manuscript again and correct typos. 


\vspace{5mm}

\noindent Comment18:
\blue{Why the energy of the level is set to zero after Eq. S 21?}

\vspace{5mm}


\noindent Reply18: We thank the Referee for pointing out this issue. The external energy level should be near zero if we want to consider the flipping of the Majorana parity. We take it to be exactly zero for the simplicity of the formula. However, it can be any other value with the results qualitatively unchanged. Please see in the below the results for several different energies.

\vspace{5mm}
\noindent Comment19:
\blue{After Eq. S22 it is said that T is the tunneling matrix. But how it
looks like and what does it describe? When T enters Eq. S23 it is
still the matrix?}
\vspace{5mm}


\noindent Reply19: We thank the Referee for the comments. We tunneling matrix is a conventional nomenclature in the study of electron tunneling problem with tunneling Hamiltonian method. In our context, it is actually a number. We change the name to 'tunneling strength' for clarity.


\vspace{5mm}
\noindent Comment20:
\blue{There is an interesting observation after Eq. S24 that the hysteresis
behavior is not altered by the parity flipping. Can this be
discussed, why this is the case?}
\vspace{5mm}


\noindent Reply20: We thank the referee for this helpful comment. The parity flipping by external levels would only flip the total parity between even and odd. However, for both even and odd total parities, the hysteresis will appear. Therefore the parity flipping will not change the hysteresis behavior. We add a discussion below Eq. (S24) to elaborate this point.




\vspace{1cm}

\noindent {\bf Summary of changes:}

(1)
The fifteenth line of the second paragraph of page 3 has been changed from

'
'

to

''

(2)
In the end of the fist paragraph on the third page, we add a sentence ""






\end{document}
